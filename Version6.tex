\documentclass[11pt]{article}

% ------------------------------------------------------------
% Packages
% ------------------------------------------------------------
\usepackage[T1]{fontenc}
\usepackage[utf8]{inputenc}
\usepackage{lmodern}
\usepackage{amsmath,amssymb,amsthm,mathtools}
\usepackage{geometry}
\usepackage{hyperref}
\geometry{margin=3cm}

% ------------------------------------------------------------
% Theorem environments
% ------------------------------------------------------------
\newtheorem{theorem}{Theorem}
\newtheorem{proposition}{Proposition}
\newtheorem{assumption}{Assumption}
\newtheorem{lemma}{Lemma}
\newtheorem{corollary}{Corollary}

\theoremstyle{remark}
\newtheorem{remark}{Remark}

\theoremstyle{definition}
\newtheorem{definition}{Definition}

% ------------------------------------------------------------
% Notation
% ------------------------------------------------------------
\newcommand{\Id}{\mathrm{Id}}
\newcommand{\Lip}{\mathrm{Lip}}
\newcommand{\Bs}{\mathcal{B}_s}
\newcommand{\Bw}{\mathcal{B}_w}
\newcommand{\norms}[1]{\left\|#1\right\|_s}
\newcommand{\normw}[1]{\left\|#1\right\|_w}

\newcommand{\cH}{\mathcal{H}} % constraint hyperplane
\newcommand{\E}{\mathcal{E}}  % kernel space

% Operator norms
\newcommand{\opnormsw}[1]{\left\|#1\right\|_{s\to w}}
\newcommand{\opnormww}[1]{\left\|#1\right\|_{w\to w}}
\newcommand{\opnormws}[1]{\left\|#1\right\|_{w\to s}}

\title{Two-norm a posteriori bounds for normalized fixed points:
bias + Newton--Kantorovich}
\author{Stefano Galatolo and Isaia Nisoli}
\date{}

%------------------------------------------------------------
% Extra macros needed by the expanded Gaussian-coupled section
%------------------------------------------------------------
\newcommand{\C}{\mathbb C}
\newcommand{\R}{\mathbb R}
\newcommand{\T}{\mathbb T}
\newcommand{\cHat}{\widehat}
\begin{document}
\maketitle

%============================================================
\section{Setting: weak--strong scale and normalized fixed points}
%============================================================

Let $\Bs\subset \Bw$ be Banach spaces with continuous embedding:
\[
\normw{u}\le \norms{u}\qquad (u\in \Bs).
\]
\noindent Here one may take $C_T:=\|\bar\rho_\sigma\|_2$, $C_J:=\|\bar\rho_\sigma'\|_2$, and $C_J^{(2)}:=\|\bar\rho_\sigma''\|_2$; see Appendix~\ref{app:gaussian-constants}.

Fix a nonzero bounded linear functional $\ell\in(\Bw)^\ast$ and define
\[
\cH:=\{f\in\Bw:\ \ell(f)=1\},\qquad
\E:=\ker\ell=\{h\in\Bw:\ \ell(h)=0\},\qquad
\E_s:=\E\cap\Bs.
\]
Let $U\subset\Bw$ be open and let $T,T_N:U\to\Bw$ be (possibly nonlinear) maps.

\begin{assumption}[Constraint invariance]\label{ass:inv}
For all $f\in U\cap\cH$ we have $\ell(T(f))=\ell(f)$ and $\ell(T_N(f))=\ell(f)$.
Equivalently, $T(\cH\cap U)\subset\cH$ and $T_N(\cH\cap U)\subset\cH$.
\end{assumption}

Define residuals
\[
\Phi(f):=f-T(f),\qquad \Phi_N(f):=f-T_N(f).
\]
By Assumption~\ref{ass:inv}, $\Phi(f),\Phi_N(f)\in\E$ whenever $f\in\cH\cap U$.

We will work on the affine slice $\tilde f_N+\E$ around a candidate $\tilde f_N\in\cH\cap U\cap \Bs$.
For $u\in\E_s$ with $\tilde f_N+u\in U$, set
\[
F(u):=\Phi(\tilde f_N+u)\in\E,\qquad
F_N(u):=\Phi_N(\tilde f_N+u)\in\E.
\]

%============================================================
\section{Two-stage a posteriori bound: bias + Newton--Kantorovich}
%============================================================

\subsection{Stage 1: NK for $T_N$ gives a contraction and a certified discrete fixed point}

Recall that we work on the constraint slice $\tilde f_N+\E$ (so $\ell(\tilde f_N+u)=1$ iff $u\in\E$),
and set
\[
F(u):=\Phi(\tilde f_N+u)=\tilde f_N+u-T(\tilde f_N+u)\in \E,\qquad
F_N(u):=\Phi_N(\tilde f_N+u)=\tilde f_N+u-T_N(\tilde f_N+u)\in \E.
\]
Assume $T_N$ preserves the constraint so that $F_N(u)\in\E$ whenever $u\in\E$.

Define the base Jacobian on $\E$:
\[
J_{N,0}:=\bigl(\Id-DT_N(\tilde f_N)\bigr)\big|_{\E}:\E\to\E,
\qquad
A_N:=J_{N,0}^{-1}:\E\to\E,
\]
whenever $J_{N,0}$ is invertible.
Define the base-preconditioned NK map on $\E$:
\[
\mathcal G_N(u):=u-A_NF_N(u).
\]
Note: $F_N(u)=0$ iff $\mathcal G_N(u)=u$, and then $f_N:=\tilde f_N+u$ satisfies $T_N(f_N)=f_N$.

\begin{assumption}[Weak Lipschitz control of $DT_N$ on a weak ball]\label{ass:lipTN-ww}
There exist $\gamma_N\ge 0$ and $R>0$ such that for all $u,v\in \E$ with
$\|u\|_w,\|v\|_w\le R$ and $\tilde f_N+u,\tilde f_N+v\in U$,
\[
\opnormww{\,DT_N(\tilde f_N+u)-DT_N(\tilde f_N+v)\,}\ \le\ \gamma_N\,\|u-v\|_w .
\]
\end{assumption}

\begin{theorem}[NK contraction for $T_N$ (weak-only form)]\label{thm:NK-TN-ww}
Assume:
\begin{enumerate}
\item $J_{N,0}$ is invertible and $M_N:=\opnormww{A_N}<\infty$.
\item Assumption~\ref{ass:lipTN-ww} holds with constants $\gamma_N,R$.
\item Residual at the base: $\delta_N:=\|F_N(0)\|_w=\|\Phi_N(\tilde f_N)\|_w$ is known.
\item Smallness:
\[
2M_N^2\,\gamma_N\,\delta_N\le 1.
\]
Define the Kantorovich radius
\[
r_N :=
\frac{1-\sqrt{1-2M_N^2\gamma_N\,\delta_N}}{M_N\gamma_N},
\qquad
(\text{interpret }r_N:=M_N\delta_N\text{ if }\gamma_N=0),
\]
and assume $r_N\le R$.
\end{enumerate}
Then $\mathcal G_N$ maps the closed ball $\overline{B_w(0,r_N)}\subset\E$ into itself and is a strict contraction there:
\[
\|\mathcal G_N(u)-\mathcal G_N(v)\|_w \le \kappa_N\,\|u-v\|_w
\qquad(u,v\in \overline{B_w(0,r_N)}),
\]
with contraction factor
\[
\kappa_N := M_N\gamma_N\,r_N = 1-\sqrt{1-2M_N^2\gamma_N\,\delta_N}\in(0,1).
\]
Consequently, $F_N(u)=0$ has a unique solution $u_N\in \overline{B_w(0,r_N)}$, and
$f_N:=\tilde f_N+u_N$ satisfies
\[
T_N(f_N)=f_N,\qquad \ell(f_N)=1,\qquad \|f_N-\tilde f_N\|_w=\|u_N\|_w\le r_N.
\]
\end{theorem}

\begin{proof}
Fix $u,v$ with $\|u\|_w,\|v\|_w\le r_N\le R$.
Using the integral remainder identity (see Lemma~\ref{lem:int-remainder} in Section~\ref{sec:relaxFrechet}),
\[
F_N(u)-F_N(v)=\int_0^1 DF_N(v+t(u-v))(u-v)\,dt,
\qquad
DF_N(w)=\Id-DT_N(\tilde f_N+w).
\]
Hence
\[
\mathcal G_N(u)-\mathcal G_N(v)
=(u-v)-A_N(F_N(u)-F_N(v))
=
A_N\int_0^1 (DF_N(v+t(u-v))-DF_N(0))(u-v)\,dt.
\]
Taking $\|\cdot\|_w$ norms and using $\|A_N\|_{w\to w}\le M_N$,
\[
\|\mathcal G_N(u)-\mathcal G_N(v)\|_w
\le
M_N\int_0^1
\|DT_N(\tilde f_N+v+t(u-v))-DT_N(\tilde f_N)\|_{w\to w}\,\|u-v\|_w\,dt.
\]
By Assumption~\ref{ass:lipTN-ww}, the operator difference is
$\le \gamma_N\|v+t(u-v)\|_w\le \gamma_N r_N$.
Thus
\[
\|\mathcal G_N(u)-\mathcal G_N(v)\|_w \le (M_N\gamma_N r_N)\,\|u-v\|_w=\kappa_N\|u-v\|_w.
\]
Ball invariance follows from the standard Kantorovich quadratic estimate:
for $\|u\|_w\le r_N$,
\[
\|\mathcal G_N(u)\|_w\le M_N\delta_N+\frac{M_N\gamma_N}{2}\|u\|_w^2
\le M_N\delta_N+\frac{M_N\gamma_N}{2}r_N^2=r_N,
\]
because $r_N$ is the smallest root of the majorant polynomial.
Banach's fixed point theorem yields the unique fixed point $u_N$ and $\|u_N\|_w\le r_N$.
\end{proof}

\subsection{Alternative CAP-method, using Krawczyk}

\begin{theorem}[Krawczyk / preconditioned Newton contraction for $T_N$]\label{thm:Krawczyk-TN}
Assume $F_N:\E\to\E$ is Fr\'echet differentiable on the weak ball
$X=\overline{B_w(0,r)}\subset \E$.
Let $A_N:\E\to\E$ be a bounded linear operator and set
$\mathcal G_N(u):=u-A_NF_N(u)$.
Assume the following quantities are known:
\begin{align*}
Y_N &:= \|A_NF_N(0)\|_w,\\
Z_N &\ge \sup_{u\in X}\ \opnormww{\Id - A_N DF_N(u)}.
\end{align*}
If $Z_N<1$ and $Y_N\le (1-Z_N)\,r$, then:
\begin{enumerate}
\item $\mathcal G_N(X)\subseteq X$;
\item $\mathcal G_N$ is a strict contraction on $X$ with factor $Z_N$:
\[
\|\mathcal G_N(u)-\mathcal G_N(v)\|_w \le Z_N\|u-v\|_w \quad (u,v\in X);
\]
\item there exists a unique $u_N\in X$ with $F_N(u_N)=0$; hence
$f_N:=\tilde f_N+u_N$ satisfies $T_N(f_N)=f_N$ and $\ell(f_N)=1$, with
$\|u_N\|_w\le r$.
\end{enumerate}
\end{theorem}

\begin{proof}
For $u,v\in X$,
\[
\mathcal G_N(u)-\mathcal G_N(v)
=\int_0^1 \bigl(\Id-A_NDF_N(v+t(u-v))\bigr)\,(u-v)\,dt.
\]
Taking $\|\cdot\|_w$ and using the definition of $Z_N$ gives the contraction bound.
Moreover,
\[
\|\mathcal G_N(u)\|_w \le \|\mathcal G_N(0)\|_w + Z_N\|u\|_w
= Y_N + Z_N\|u\|_w \le Y_N+Z_N r \le r,
\]
so $\mathcal G_N(X)\subseteq X$.
Banach's theorem yields the unique fixed point $u_N\in X$, which satisfies $F_N(u_N)=0$.
\end{proof}


%------------------------------------------------------------
\subsection{Stage 2: discretization bias $\|f-f_N\|_w$ from the NK contraction
(using a uniform Lasota--Yorke bound for $T$)}
%------------------------------------------------------------

Recall the base-preconditioned maps (same preconditioner $A_N$):
\[
\mathcal G_N(u):=u-A_NF_N(u),\qquad
\mathcal G(u):=u-A_NF(u),
\]
where $F_N(u)=\Phi_N(\tilde f_N+u)$ and $F(u)=\Phi(\tilde f_N+u)$.

Let
\[
D:=\overline{B_w(0,r_N)}\cap\E
\]
be the NK ball from Theorem~\ref{thm:NK-TN-ww}, and let $u_N\in D$ be the unique fixed point of
$\mathcal G_N$ (equivalently $f_N:=\tilde f_N+u_N$ is the unique fixed point of $T_N$ in $\tilde f_N+D$).

%------------------------------------------------------------
\subsubsection*{(A) Pointwise telescoping (no $\sup_{u\in D}$ required)}
%------------------------------------------------------------

\begin{lemma}[Pointwise telescoping bound]\label{lem:telescope-pointwise}
Assume:
\begin{enumerate}
\item $\mathcal G_N:D\to D$ is a contraction in $\|\cdot\|_w$ with constant $\kappa_N\in[0,1)$
and has a fixed point $u_N\in D$;
\item $\mathcal G$ has a fixed point $u\in D$.
\end{enumerate}
Then
\[
\|u-u_N\|_w\ \le\ \frac{\|\mathcal G(u)-\mathcal G_N(u)\|_w}{1-\kappa_N}.
\]
\end{lemma}

\begin{proof}
Using $\mathcal G(u)=u$ and $\mathcal G_N(u_N)=u_N$,
\[
\|u-u_N\|_w=\|\mathcal G(u)-\mathcal G_N(u_N)\|_w
\le \|\mathcal G(u)-\mathcal G_N(u)\|_w + \|\mathcal G_N(u)-\mathcal G_N(u_N)\|_w
\le \|\mathcal G(u)-\mathcal G_N(u)\|_w + \kappa_N\|u-u_N\|_w.
\]
Rearrange.
\end{proof}

%------------------------------------------------------------
\subsubsection*{(B) $s\to w$ discretization control for the map}
%------------------------------------------------------------

\begin{assumption}[$s\to w$ discretization bound for $T-T_N$]\label{ass:map-sw}
There exists a constant $\varepsilon_N\ge 0$ such that for all $g$ in the relevant domain
(with $g\in\Bs$),
\[
\|T(g)-T_N(g)\|_w \ \le\ \varepsilon_N\,\|g\|_s.
\]
Equivalently, $\varepsilon_N=\|T-T_N\|_{s\to w}$ on that domain.
\end{assumption}

\begin{lemma}[Mismatch at the true fixed point]\label{lem:mismatch-at-true-fp}
Assume Assumption~\ref{ass:map-sw}. Let $u\in D$ be a fixed point of $\mathcal G$ and set
$f:=\tilde f_N+u$. Then
\[
\|\mathcal G(u)-\mathcal G_N(u)\|_w
=
\bigl\|A_N\bigl(T(f)-T_N(f)\bigr)\bigr\|_w
\ \le\
\opnormww{A_N}\,\varepsilon_N\,\|f\|_s.
\]
\end{lemma}

\begin{proof}
Since $\mathcal G(u)-\mathcal G_N(u)=A_N\bigl(T(\tilde f_N+u)-T_N(\tilde f_N+u)\bigr)$,
apply the operator norm bound and Assumption~\ref{ass:map-sw}.
\end{proof}

%------------------------------------------------------------
\subsubsection*{(C) Uniform Lasota--Yorke for $T$ yields a strong bound for fixed points}
%------------------------------------------------------------

\begin{assumption}[Uniform Lasota--Yorke for $T$ on the invariant slice]\label{ass:LY-T-uniform}
There exist constants $a\in[0,1)$ and $b\ge 0$ such that for all $g$ in the relevant domain
(e.g.\ $g\in U\cap\cH$ and $T(g)\in\Bs$),
\[
\|T(g)\|_s \le a\,\|g\|_s + b\,\|g\|_w.
\]
\end{assumption}

\begin{lemma}[Fixed point strong bound from Lasota--Yorke]\label{lem:LY-fp-strong}
Assume Assumption~\ref{ass:LY-T-uniform}.
If $f$ is a fixed point of $T$ (i.e.\ $T(f)=f$) and $f\in\Bs$, then
\[
\|f\|_s \le \frac{b}{1-a}\,\|f\|_w.
\]
If moreover $\|f\|_w\le W$ is known a priori on the invariant slice, then $\|f\|_s\le \frac{b}{1-a}W$.
\end{lemma}

\begin{proof}
From $f=T(f)$ and the Lasota--Yorke inequality:
\[
\|f\|_s=\|T(f)\|_s\le a\|f\|_s+b\|f\|_w,
\]
hence $(1-a)\|f\|_s\le b\|f\|_w$.
\end{proof}

\begin{remark}[Why this is the right hypothesis in the coupled/Gaussian setting]
In many self-consistent transfer operator couplings (including the Galatolo--Tanzi framework
and Gaussian-noise regularization), Assumption~\ref{ass:LY-T-uniform} holds uniformly on the invariant slice,
and the fixed point is automatically in $\Bs$ (regularization). In that case, the lemma applies
without any ``local bridge'' assumption and without using any finite-dimensional constant $B_N$.
\end{remark}

%------------------------------------------------------------
\subsubsection*{(D) Bias theorem (no $B_N$ anywhere in Stage 2)}
%------------------------------------------------------------

\begin{theorem}[Bias bound $\|f-f_N\|_w$ using $s\to w$ discretization + Lasota--Yorke]\label{thm:bias-LY}
Assume Theorem~\ref{thm:NK-TN-ww} and let $u_N\in D$ be the unique fixed point of $\mathcal G_N$.
Assume that $T$ has a normalized fixed point $f$ with $u:=f-\tilde f_N\in D$.
Assume Assumptions~\ref{ass:map-sw} and \ref{ass:LY-T-uniform}, and suppose an a priori weak bound
$\|f\|_w\le W$ is available on the invariant slice.
Then
\[
\|f-f_N\|_w=\|u-u_N\|_w
\ \le\
\frac{1}{1-\kappa_N}\,\opnormww{A_N}\,\varepsilon_N\,\|f\|_s
\ \le\
\frac{1}{1-\kappa_N}\,\opnormww{A_N}\,\varepsilon_N\,\frac{b}{1-a}\,W.
\]
\end{theorem}

\begin{proof}
Combine Lemma~\ref{lem:telescope-pointwise} with Lemma~\ref{lem:mismatch-at-true-fp}, then apply
Lemma~\ref{lem:LY-fp-strong}.
\end{proof}

\begin{corollary}[Two-stage a posteriori bound to $\tilde f_N$]\label{cor:two-stage-LY}
Under the hypotheses of Theorem~\ref{thm:NK-TN-ww} and Theorem~\ref{thm:bias-LY},
\[
\|f-\tilde f_N\|_w
\le
\|f-f_N\|_w+\|f_N-\tilde f_N\|_w
\le
\frac{\opnormww{A_N}}{1-\kappa_N}\,\varepsilon_N\,\frac{b}{1-a}\,W
\;+\; r_N.
\]
\end{corollary}

\begin{remark}[About $W$]
In many applications $W$ is explicit from normalization/positivity (e.g.\ if $\|\cdot\|_w=\|\cdot\|_{L^1}$
and $\ell(f)=\int f=1$, then $W=1$). In other weak norms (e.g.\ $L^2$), one needs a separate a priori
bound on $\|f\|_w$ on the invariant slice; this is independent of any finite-dimensional bridge constant.
\end{remark}


%============================================================
\section{Application: coupled random dynamical systems with Gaussian noise}
\label{sec:gaussian-coupled}
%============================================================

\subsection{General mean-field coupling on $\mathbb T$}

Let $\mathbb T=\mathbb R/\mathbb Z$ and let $T_0:\mathbb T\to\mathbb T$ be a measurable map.
Let $\bar\rho_\sigma$ be the $1$-periodized Gaussian kernel of variance $\sigma^2>0$,
so $\bar\rho_\sigma\ge 0$ and $\int_{\mathbb T}\bar\rho_\sigma=1$.

For $c\in\mathbb T$, define the noisy transfer operator
\[
(P_c f)(x):=\int_{\mathbb T}\bar\rho_\sigma\!\bigl(x-T_0(y)-c\bigr)\,f(y)\,dy.
\]
This is the annealed transfer operator for the RDS
\[
X_{n+1}=T_0(X_n)+c+\sigma\,\xi_n \pmod 1,
\qquad \xi_n\sim\mathcal N(0,1)\ \text{i.i.d.}
\]

We introduce a self-consistent (mean-field) coupling by setting
\[
m(f):=\langle \phi,f\rangle=\int_{\mathbb T}\phi(x)\,f(x)\,dx,
\qquad
c(f):=\delta\,G(m(f)),
\]
where $\phi:\mathbb T\to\mathbb R$ is an observable, $G:\mathbb R\to\mathbb R$ is Lipschitz,
and $\delta\in\mathbb R$ is the coupling strength.
Define the self-consistent map
\[
T(f):=P_{c(f)}f.
\]
A normalized fixed point $f^\ast$ (with $\ell(f^\ast)=\int_{\mathbb T} f^\ast=1$) is a stationary law
for the coupled RDS
\[
X_{n+1}=T_0(X_n)+\delta\,G\!\bigl(m(f_n)\bigr)+\sigma\,\xi_n \pmod 1,
\qquad f_{n+1}=T(f_n).
\]

\subsection{Two norms: analytic strong space and $L^2$ weak space}

Fix $\tau>0$ and let $\mathcal A_\tau$ be the analytic strip space on $\mathbb T$
(e.g.\ Fourier-weighted $\ell^2$ with weight $e^{2\pi\tau|k|}$), and let $L^2$ be the weak space.
Write
\[
\|\,\cdot\,\|_s:=\|\,\cdot\,\|_{\mathcal A_\tau},\qquad \|\,\cdot\,\|_w:=\|\,\cdot\,\|_2.
\]

Gaussian noise gives a smoothing estimate (uniform in $c$):
\[
\|P_c f\|_{\mathcal A_\tau}\ \le\ S_{\tau,\sigma}\,\|f\|_2,
\qquad
\|\partial_c P_c f\|_{\mathcal A_\tau}\ \le\ S^{(1)}_{\tau,\sigma}\,\|f\|_2,
\qquad
\|\partial_c^2 P_c f\|_{\mathcal A_\tau}\ \le\ S^{(2)}_{\tau,\sigma}\,\|f\|_2,
\]
with explicit constants $S_{\tau,\sigma},S^{(1)}_{\tau,\sigma},S^{(2)}_{\tau,\sigma}$ (Fourier multipliers).

Moreover, $P_c$ is Markov on $L^1$ and satisfies a weak bound on densities:
for $\|f\|_1=1$,
\[
\|P_c f\|_\infty\le \|\bar\rho_\sigma\|_\infty,
\qquad
\|P_c f\|_2\le \|\bar\rho_\sigma\|_2.
\]
Hence any normalized fixed point $f^\ast=T(f^\ast)$ satisfies $\|f^\ast\|_2\le \|\bar\rho_\sigma\|_2$,
which is a convenient choice of the weak constant $W$ in Assumption~\ref{ass:LY-T-uniform}.

\subsection{Fourier/Galerkin discretization}

Let $\Pi_N$ be the Fourier projection onto modes $|k|\le N$ and let $\mathcal V_N:=\mathrm{Ran}(\Pi_N)$.
Define the discrete surrogate (any of the equivalent variants you use in the note can be plugged here):
\[
T_N(f):=\Pi_N\,P_{c(\Pi_N f)}\,\Pi_N f,
\qquad f\in \mathcal V_N.
\]
Compute a numerical candidate $\tilde f\in\mathcal V_N$ with $\ell(\tilde f)=1$,
typically by a Newton solve in finite dimension.

\subsection{A posteriori certification: what must be computed}

The derivative has the explicit structure (rank-one update + translated transfer operator):
\[
DT(f)h = P_{c(f)}h \;+\; c'(f)[h]\;\partial_cP_{c(f)}f,
\qquad
c'(f)[h]=\delta\,G'(m(f))\,\langle \phi,h\rangle.
\]
The same form holds for $DT_N(\tilde f)$ after projecting to $\mathcal V_N$,
so the Jacobian on $\ker\ell$ is a finite matrix (plus rank-one), suitable for rigorous inversion.


\begin{corollary}[Rigorous fixed point enclosure for the coupled Gaussian RDS: checklist]\label{cor:gaussian-checklist}
Fix parameters $(\sigma,\tau,\delta)$ and a coupling $(G,\phi)$ as above.
Assume the uniform Lasota--Yorke bounds \emph{for the coupled operator} (hence in particular for every $c\in\mathbb R$):
\[
\|T(f)\|_2\le C_T\,\|f\|_2,
\qquad
\|T(f)\|_{\mathcal A_\tau}\le \alpha\,\|f\|_{\mathcal A_\tau}+\beta\,\|f\|_2
\qquad(\alpha\in[0,1),\ \beta\ge 0),
\]
and the Gaussian smoothing bounds (uniform in $c$):
\[
\|P_c u\|_{\mathcal A_\tau}\le S_{\tau,\sigma}\,\|u\|_2,\qquad
\|\partial_c P_c u\|_{\mathcal A_\tau}\le S^{(1)}_{\tau,\sigma}\,\|u\|_2,
\]
with weak $L^2$ bounds
\[
\|P_c u\|_2\le C_T\,\|u\|_2,\qquad
\|\partial_c P_c u\|_2\le C_J\,\|u\|_2,\qquad
\|\partial_c^2 P_c u\|_2\le C_J^{(2)}\,\|u\|_2.
\]
Let $\Pi_N$ be the Fourier truncation and define
\[
T_N(f):=\Pi_N\,T(\Pi_N f)
\qquad\text{(so in particular the coupling uses } m(\Pi_N f)\text{)}.
\]

Let $\tilde f_N\in \mathcal A_\tau\cap\cH$ be a numerically computed approximate fixed point of $T_N$, and set
\[
K_2:=\|\tilde f_N\|_2,\qquad K_\tau:=\|\tilde f_N\|_{\mathcal A_\tau},\qquad
\delta_N:=\|T_N(\tilde f_N)-\tilde f_N\|_2.
\]
Write the coupling regularity constants as
\[
L_G:=\|G'\|_\infty,\qquad \mathrm{Lip}(G):=\sup_{x\ne y}\frac{|G(x)-G(y)|}{|x-y|},\qquad
L_{G'}:=\mathrm{Lip}(G')\ \ (\text{e.g. }L_{G'}=\|G''\|_\infty\text{ if }G\in C^2).
\]

Assume you have a certified weak inverse bound on the discrete Jacobian on the constraint slice:
\[
M_N\ \ge\ \bigl\|(I-DT_N(\tilde f_N))^{-1}\|_{2\to 2;\ker\ell}.
\]

Define the explicit truncation and mismatch errors (value level)
\[
e_T:=e^{-2\pi\tau N}(S_{\tau,\sigma}+1)K_\tau,
\qquad
e_{\mathrm{mis}}:=|\delta|\,\mathrm{Lip}(G)\,e^{-2\pi\tau N}\,\|\phi\|_{\mathcal A_\tau}\,C_J\,K_2^2,
\]
and set the total defect
\[
\Delta := \|T(\tilde f_N)-\tilde f_N\|_2\ \le\ \delta_N + e_T + e_{\mathrm{mis}}.
\]
(In particular, if $\phi$ is a trigonometric polynomial of degree $\le N$, then $m(\Pi_N f)=m(f)$ and $e_{\mathrm{mis}}=0$.)

Define the explicit Jacobian mismatch (strong$\to$weak)
\[
\varepsilon_J:=\|DT(\tilde f_N)-DT_N(\tilde f_N)\|_{\mathcal A_\tau\to L^2},
\]
and bound it by
\[
\varepsilon_J\ \le\ e^{-2\pi\tau N}(S_{\tau,\sigma}+1)
+|\delta|L_G\|\phi\|_2K_\tau e^{-2\pi\tau N}\bigl(S^{(1)}_{\tau,\sigma}+C_J\bigr)
+e^{-2\pi\tau N}C_{\mathrm{mis}}(K_2),
\]
where
\[
C_{\mathrm{mis}}(K_2):=
|\delta|\,\|\phi\|_{\mathcal A_\tau}\Bigl[(\mathrm{Lip}(G)+L_G)C_JK_2
+L_{G'}\|\phi\|_2C_JK_2^2\Bigr]
+|\delta|^2 L_G\mathrm{Lip}(G)\,\|\phi\|_2\,\|\phi\|_{\mathcal A_\tau}\,C_J^{(2)}K_2^2.
\]
(Again, if $\deg(\phi)\le N$ then $C_{\mathrm{mis}}(K_2)=0$.)

Assume a certified inverse bound
\[
M:=\bigl\|(I-DT(\tilde f_N))^{-1}\bigr\|_{2\to 2;\ker\ell}<\infty.
\]
(For instance, if a reference inverse bound $M_N$ and a Jacobian mismatch estimate $\varepsilon_J$ satisfy $M_N\varepsilon_J<1$, then the Neumann-series argument gives $M\le \frac{M_N}{1-M_N\varepsilon_J}$.)

Finally, define the explicit Lipschitz constant for the differential on an $L^2$-ball $\{f:\|f\|_2\le K\}$ by
\[
\|DT(f)-DT(g)\|_{2\to 2}\ \le\ \gamma(K)\,\|f-g\|_2,
\]
with
\[
\gamma(K):=
|\delta|\,\|\phi\|_2\,C_J\bigl(\mathrm{Lip}(G)+L_G\bigr)
+|\delta|\,L_{G'}\,\|\phi\|_2^2\,C_JK
+|\delta|^2 L_G\mathrm{Lip}(G)\,\|\phi\|_2^2\,C_J^{(2)}K.
\]

Choose any $K\ge K_2$ and set $h:=M^2\gamma(K)\Delta$.
If $h\le\tfrac12$, then there exists a unique normalized fixed point $f^\ast$ of $T$ in the $L^2$-ball $B_r(\tilde f_N)$, and
\[
\|f^\ast-\tilde f_N\|_2\le r,
\qquad
r:=\frac{1-\sqrt{1-2h}}{M\gamma(K)}
\quad(\text{interpret }r:=M\Delta\text{ if }\gamma(K)=0).
\]
Moreover, the Newton map on $\ell(f)=1$ is a contraction on $B_r(\tilde f_N)$ with contraction constant
\[
\kappa=\frac{1-\sqrt{1-2h}}{h}\in(0,1).
\]
\end{corollary}


\begin{remark}
The point of Corollary~\ref{cor:gaussian-checklist} is that \emph{all hard quantities}
($M_N$ and $\delta_N$) are computed in finite dimension, while the infinite-dimensional
errors enter only through explicit $e^{-2\pi\tau N}$ factors and the Gaussian smoothing constants.
No global bridge constant $B_N$ is used outside finite dimension.
\end{remark}


%============================================================
\section{Example: unimodal map (quadratic/logistic) modulo $1$}
\label{sec:unimodal-example}
%============================================================

Let $T_0(x)=a\,x(1-x)\pmod 1$ on $\mathbb T$ (with $a\in(0,4]$), and keep the same Gaussian noise and coupling
$c(f)=\delta G(\langle\phi,f\rangle)$.

In the Fourier basis $\{e^{2\pi i k x}\}$, the operator $P_c$ has matrix elements
\[
\widehat{P_c f}(k)=\widehat{\bar\rho_\sigma}(k)\,e^{-2\pi i k c}\,
\sum_{m\in\mathbb Z} V_k(m)\,\hat f(m),
\qquad
V_k(m):=\int_{\mathbb T} e^{-2\pi i k T_0(y)}\,e^{2\pi i m y}\,dy,
\]
and $\widehat{\bar\rho_\sigma}(k)=e^{-2\pi^2\sigma^2 k^2}$.
The only map-dependent ingredient is the (rigorously computed) oscillatory integral $V_k(m)$.
Since the integrand is smooth and periodic, one may compute $V_k(m)$ by a validated trapezoidal rule
or any rigorous quadrature scheme; the resulting bounds propagate directly into the finite-dimensional
matrix for $P_c$ restricted to $\mathcal V_N$.

\begin{corollary}[Concrete workflow for the unimodal example]\label{cor:unimodal-workflow}
Fix parameters $(a,\sigma,\delta,\tau)$ and choose $N$.
To obtain a rigorous enclosure of the unique normalized fixed point $f^\ast$ of the coupled system:
\begin{enumerate}
\item Compute a numerical candidate $\tilde f\in\mathcal V_N$ and enforce $\ell(\tilde f)=1$.
\item Build the $(2N{+}1)\times(2N{+}1)$ matrix for $P_{c(\tilde f)}$ on modes $|k|\le N$ using
validated approximations of $V_k(m)$ and the exact multiplier $e^{-2\pi^2\sigma^2 k^2}$.
\item Assemble $DT_N(\tilde f)$ using the rank-one structure from $c'(f)[h]$ and $\partial_cP_c f$,
and rigorously bound $M_N=\|(I-DT_N(\tilde f))^{-1}\|_{2\to 2}$ on $\ker\ell\cap\mathcal V_N$.
\item Compute $\delta_N=\|T_N(\tilde f)-\tilde f\|_2$.
\item Compute the explicit analytic constants entering $e_T$, $\varepsilon_J$, and $\gamma(K)$
(using $\|\phi\|_2$, $\|\phi\|_{\mathcal A_\tau}$, $\mathrm{Lip}(G)$, bounds on $G'$ and $G''$,
and $S_{\tau,\sigma},S^{(1)}_{\tau,\sigma},S^{(2)}_{\tau,\sigma}$).
\item Check the inequalities of Corollary~\ref{cor:gaussian-checklist}
(Neumann test and Kantorovich smallness). If they hold, conclude a unique fixed point $f^\ast$
with an explicit radius $r$ such that $\|f^\ast-\tilde f\|_2\le r$.
\end{enumerate}
\end{corollary}

\section{Appendix: Tapering and smoothing a map for FFT-based discretization}

\paragraph{Goal.}
Let $T:[-1,1]\to\mathbb R$ be a (non-periodic) map. In order to build an FFT-based
finite-rank approximation of the annealed transfer operator with Gaussian noise,
we construct from $T$ a periodic, FFT-friendly surrogate $\widetilde T$ on the
period-$2$ torus $\mathbb T_2\simeq[-1,1]$ by a (i) tapering step near the boundary,
followed by an optional (ii) Gaussian smoothing step, and (iii) Fourier truncation.
We provide explicit bounds for $\|T-\widetilde T\|_\infty$ and for the induced operator error.

\begin{definition}[Taper window]\label{def:taper-window}
Fix a collar width $\eta\in(0,1)$ and a smooth step function $s:[0,1]\to[0,1]$
satisfying $s(0)=0$, $s(1)=1$. Define an even taper window $w_\eta:[-1,1]\to[0,1]$ by
\[
w_\eta(x):=
\begin{cases}
1, & |x|\le 1-\eta,\\[2mm]
s\!\left(\dfrac{1-|x|}{\eta}\right), & 1-\eta<|x|<1.
\end{cases}
\]
Assume $\|s'\|_\infty\le C_1$ so that $\|w_\eta'\|_\infty\le C_1/\eta$.
\end{definition}

\begin{definition}[Derivative-tapered periodic map]\label{def:Teta}
Assume $T\in C^2([-1,1])$. Define $T_\eta$ (up to an additive constant) by tapering the derivative:
\[
T_\eta'(x):=w_\eta(x)\,T'(x),\qquad x\in[-1,1],
\]
and fix the additive constant by requiring $T_\eta(0)=T(0)$.
Then $T_\eta$ agrees with $T$ on $[-1+\eta,\,1-\eta]$ up to an additive constant and is
flat at $\pm1$ in the sense that $T_\eta'(\pm1)=0$.
We regard $T_\eta$ as a periodic function on $\mathbb T_2$.
\end{definition}

\begin{definition}[Period-$2$ Gaussian smoothing and Fourier truncation]\label{def:smooth-trunc}
Let $\bar\rho_{\sigma_{\rm sm}}$ be the period-$2$ Gaussian kernel with Fourier multiplier
\[
\widehat{\bar\rho_{\sigma_{\rm sm}}}(k)=\exp\!\Bigl(-\tfrac{\pi^2\sigma_{\rm sm}^2}{2}\,k^2\Bigr),
\qquad k\in\mathbb Z.
\]
Define the smoothing operator $G_{\sigma_{\rm sm}}f:=\bar\rho_{\sigma_{\rm sm}}*f$ on $\mathbb T_2$.
Let $\Pi_N$ denote the Fourier projector onto modes $|k|\le N$ (period-$2$ convention).
Set
\[
\widetilde T:=\Pi_N\bigl(G_{\sigma_{\rm sm}}T_\eta\bigr).
\]
\end{definition}

\begin{lemma}[Explicit sup-norm bound for $T-\widetilde T$]\label{lem:T-Ttilde}
With $\widetilde T$ as in Definition~\ref{def:smooth-trunc},
\[
\|T-\widetilde T\|_\infty
\le
\underbrace{\|T-T_\eta\|_\infty}_{\text{taper}}
+\underbrace{\|T_\eta-G_{\sigma_{\rm sm}}T_\eta\|_\infty}_{\text{smoothing}}
+\underbrace{\|G_{\sigma_{\rm sm}}T_\eta-\Pi_NG_{\sigma_{\rm sm}}T_\eta\|_\infty}_{\text{Fourier tail}}.
\]
Moreover, the following explicit bounds hold:
\begin{align*}
\|T-T_\eta\|_\infty
&\le \eta\,\|T'\|_\infty,\\[1mm]
\|T_\eta-G_{\sigma_{\rm sm}}T_\eta\|_\infty
&\le \frac{\sigma_{\rm sm}^2}{2}\,\|T_\eta''\|_\infty
\le \frac{\sigma_{\rm sm}^2}{2}\Bigl(\|T''\|_\infty+\frac{C_1}{\eta}\|T'\|_\infty\Bigr),\\[1mm]
\|G_{\sigma_{\rm sm}}T_\eta-\Pi_NG_{\sigma_{\rm sm}}T_\eta\|_\infty
&\le \frac{2\|T_\eta\|_\infty}{\pi^2\sigma_{\rm sm}^2\,N}
\exp\!\Bigl(-\tfrac{\pi^2\sigma_{\rm sm}^2}{2}\,N^2\Bigr).
\end{align*}
\end{lemma}

\begin{proof}
The triangle inequality yields the three-term decomposition.
For the taper term, note that $T_\eta'=w_\eta T'$ and $0\le w_\eta\le1$, hence
$|T(x)-T_\eta(x)|\le\int_{1-\eta}^1|T'(t)|dt\le \eta\|T'\|_\infty$ (and similarly near $-1$),
giving $\|T-T_\eta\|_\infty\le \eta\|T'\|_\infty$.

For the smoothing term, Gaussian convolution on the torus satisfies the standard second-order bound
$\|f-G_\sigma f\|_\infty\le \frac{\sigma^2}{2}\|f''\|_\infty$ for $f\in C^2$.
Differentiating $T_\eta'=w_\eta T'$ gives
$T_\eta''=w_\eta'T'+w_\eta T''$, hence
$\|T_\eta''\|_\infty\le \|w_\eta'\|_\infty\|T'\|_\infty+\|T''\|_\infty
\le \frac{C_1}{\eta}\|T'\|_\infty+\|T''\|_\infty$.

For the Fourier tail, use the Fourier multiplier formula for $G_{\sigma_{\rm sm}}$ and a Gaussian
tail estimate for $\sum_{|k|>N}e^{-c k^2}$ to obtain
$\|G_{\sigma_{\rm sm}}T_\eta-\Pi_NG_{\sigma_{\rm sm}}T_\eta\|_\infty
\le \sum_{|k|>N}|\widehat{\bar\rho_{\sigma_{\rm sm}}}(k)|\,|\widehat{T_\eta}(k)|
\le \|T_\eta\|_\infty\sum_{|k|>N}e^{-(\pi^2\sigma_{\rm sm}^2/2)k^2}$,
and then bound the tail by an integral.
\end{proof}

\paragraph{Induced perturbation bound for the annealed transfer operator.}
Let $\sigma_{\rm rds}>0$ and define the annealed transfer operator
\[
(\mathcal P_T f)(x)=\int_{-1}^1 \rho_{\sigma_{\rm rds}}(x-T(y))\,f(y)\,dy,
\qquad f\in L^1([-1,1]),
\]
where $\rho_{\sigma_{\rm rds}}$ is the (periodized) Gaussian density.
Then $\mathcal P_T:L^1\to L^1$ is bounded and depends Lipschitzly on the map in sup norm:

\begin{lemma}[$L^1\to L^1$ stability w.r.t.\ map perturbations]\label{lem:PT-stability}
Assume $\rho_{\sigma_{\rm rds}}$ is Gaussian. Then
\[
\|\mathcal P_T-\mathcal P_{\widetilde T}\|_{L^1\to L^1}
\le \frac{\sqrt{2/\pi}}{\sigma_{\rm rds}}\;\|T-\widetilde T\|_\infty.
\]
\end{lemma}

\begin{proof}
For each $y$,
$\int | \rho_{\sigma_{\rm rds}}(\cdot-T(y))-\rho_{\sigma_{\rm rds}}(\cdot-\widetilde T(y))|\,dx
\le |T(y)-\widetilde T(y)|\int|\rho_{\sigma_{\rm rds}}'(x)|dx
=|T(y)-\widetilde T(y)|\sqrt{2/\pi}/\sigma_{\rm rds}$.
Integrate in $y$ against $|f(y)|$ and take the supremum over $\|f\|_1=1$.
\end{proof}

%============================================================
% L^2 stability bound + smoothing bound (with and without C_1)
%============================================================

\paragraph{$L^2\to L^2$ stability of the annealed operator w.r.t.\ map perturbations.}
Let $\mathbb T_2\simeq[-1,1]$ be the period-$2$ torus, and let $\bar\rho_\sigma$ be the
periodized Gaussian kernel. For a measurable map $T:\mathbb T_2\to\mathbb R$ define
\[
(\mathcal P_T f)(x):=\int_{\mathbb T_2}\bar\rho_{\sigma_{\rm rds}}\!\bigl(x-T(y)\bigr)\,f(y)\,dy,
\qquad f\in L^2(\mathbb T_2).
\]
Then for any two maps $T,\widetilde T$,
\begin{lemma}[$L^2$ Lipschitz bound]\label{lem:P_L2_stability}
\[
\|\mathcal P_T-\mathcal P_{\widetilde T}\|_{L^2\to L^2}
\;\le\;
\sqrt{2}\,\|\bar\rho_{\sigma_{\rm rds}}'\|_{L^2(\mathbb T_2)}\;\|T-\widetilde T\|_{L^\infty(\mathbb T_2)}.
\]
Moreover, with the period-$2$ Fourier basis $e_k(x)=e^{i\pi kx}$ one has the explicit identity
\[
\|\bar\rho_{\sigma}'\|_{L^2(\mathbb T_2)}^2
=
\sum_{k\in\mathbb Z}(\pi k)^2
\exp\!\bigl(-\pi^2\sigma^2 k^2\bigr).
\]
\end{lemma}
\begin{proof}
For each $y$ and $u,v\in\mathbb R$, by the fundamental theorem of calculus in the shift parameter,
\[
\bar\rho(\cdot-u)-\bar\rho(\cdot-v)=\int_v^u -\bar\rho'(\cdot-t)\,dt,
\]
hence $\|\bar\rho(\cdot-u)-\bar\rho(\cdot-v)\|_2\le |u-v|\,\|\bar\rho'\|_2$.
Therefore
\[
\|(\mathcal P_T-\mathcal P_{\widetilde T})f\|_2
\le
\int_{\mathbb T_2}\|\bar\rho(\cdot-T(y))-\bar\rho(\cdot-\widetilde T(y))\|_2\,|f(y)|\,dy
\le
\|\bar\rho'\|_2\,\|T-\widetilde T\|_\infty\,\|f\|_1.
\]
Finally, $\|f\|_1\le |\mathbb T_2|^{1/2}\|f\|_2=\sqrt{2}\|f\|_2$ gives the stated $L^2\to L^2$ bound.
The Fourier identity follows from Parseval and the fact that $\widehat{\bar\rho_\sigma}(k)
=\exp(-\tfrac{\pi^2\sigma^2}{2}k^2)$, so $\widehat{\bar\rho_\sigma'}(k)=i\pi k\,\widehat{\bar\rho_\sigma}(k)$.
\end{proof}

\paragraph{Gaussian smoothing bounds in $\|\cdot\|_\infty$.}
Let $G_{\sigma_{\rm sm}}$ denote period-$2$ Gaussian convolution on $\mathbb T_2$.

\begin{lemma}[Smoothing bounds]\label{lem:smoothing_bounds}
For $f\in W^{1,\infty}(\mathbb T_2)$,
\[
\|f-G_{\sigma_{\rm sm}}f\|_\infty
\le
\sqrt{\frac{2}{\pi}}\,\sigma_{\rm sm}\,\|f'\|_\infty.
\]
If moreover $f\in W^{2,\infty}(\mathbb T_2)$, then
\[
\|f-G_{\sigma_{\rm sm}}f\|_\infty
\le
\frac{\sigma_{\rm sm}^2}{2}\,\|f''\|_\infty.
\]
\end{lemma}
\begin{proof}
Write $(G_\sigma f)(x)=\int f(x-t)\rho_\sigma(t)\,dt$. Then
\[
f(x)-G_\sigma f(x)=\int \bigl(f(x)-f(x-t)\bigr)\rho_\sigma(t)\,dt.
\]
For $W^{1,\infty}$ use $|f(x)-f(x-t)|\le |t|\|f'\|_\infty$ and $\int |t|\rho_\sigma(t)\,dt
=\sigma\sqrt{2/\pi}$. For $W^{2,\infty}$ use the second-order Taylor remainder
$|f(x)-f(x-t)-t f'(x)|\le \tfrac{t^2}{2}\|f''\|_\infty$ and $\int t^2\rho_\sigma(t)\,dt=\sigma^2$.
(The same constants hold for the periodized kernel.)
\end{proof}

\paragraph{Explicit $C_1$ for the basic bump $b(u)=e^{-1/u}$.}
Define $b(u)=e^{-1/u}$ for $u>0$ and $b(0)=0$. Then $b'(u)=e^{-1/u}/u^2$ and
\[
\sup_{u\in(0,1]} |b'(u)| = b'\!\left(\tfrac12\right)=4e^{-2}.
\]
In particular, if a taper window is built from $b$ by rescaling $u=(1-|x|)/\eta$,
one obtains a bound $\|w_\eta'\|_\infty\le (4e^{-2})/\eta$.


%============================================================
\section{Appendix: Relaxing Fr\'echet differentiability}\label{sec:relaxFrechet}
%============================================================

For the NK estimates above we only need an \emph{integral remainder formula}
and an estimate in the mixed operator norm $\|\cdot\|_{s\to w}$.

\begin{assumption}[Continuous G\^ateaux differentiability in strong directions]\label{ass:Gateaux}
For each $g\in U\cap\Bs$ there exists a bounded linear map $DT(g)\in\mathcal L(\Bs,\Bw)$ such that:
\begin{enumerate}
\item For every $h\in\Bs$,
\[
\lim_{\varepsilon\to 0}\left\|\frac{T(g+\varepsilon h)-T(g)}{\varepsilon}-DT(g)h\right\|_w=0.
\]
\item For every segment $g_t=f+th$ contained in $U\cap\Bs$,
the map $t\mapsto DT(g_t)$ is continuous in operator norm $\|\cdot\|_{s\to w}$.
\end{enumerate}
\end{assumption}

\begin{lemma}[Integral remainder formula]\label{lem:int-remainder}
Assume Assumption~\ref{ass:Gateaux}. If $f\in U\cap\Bs$ and $h\in\Bs$ are such that
$f+th\in U$ for all $t\in[0,1]$, then
\[
T(f+h)-T(f)=\int_0^1 DT(f+th)\,h\,dt
\quad\text{(Bochner integral in $\Bw$)}.
\]
Consequently,
\[
T(f+h)-T(f)-DT(f)h=\int_0^1 \bigl(DT(f+th)-DT(f)\bigr)\,h\,dt,
\]
and hence
\[
\|T(f+h)-T(f)-DT(f)h\|_w
\le
\Bigl(\sup_{t\in[0,1]}\|DT(f+th)-DT(f)\|_{s\to w}\Bigr)\,\|h\|_s.
\]
\end{lemma}

\begin{proof}
Define $\phi(t)=T(f+th)\in\Bw$.
Assumption~\ref{ass:Gateaux}(1) gives differentiability with $\phi'(t)=DT(f+th)h$,
and Assumption~\ref{ass:Gateaux}(2) gives continuity of $\phi'$, hence Bochner integrability.
Apply the Banach-space fundamental theorem of calculus.
\end{proof}

\begin{remark}[Compatibility with Galatolo--Tanzi style]
This is the exact weakening typically used when one controls the differential only as
$\|DT(\cdot)\|_{s\to w}$ and continuity is available in a weaker topology:
you never need a full Fr\'echet remainder in $\|\cdot\|_s$, only the integral formula above.
\end{remark}

\section{Appendix: Proofs of the Gaussian constants ($C_T$, $C_J$, $\gamma$)}\label{app:gaussian-constants}


% ============================================================
% Analytic strong norm + Fourier discretization for
% self-consistent noisy transfer operators with analytic coupling.
% Strong space: A_\tau, weak space: L^2.
% ============================================================

\subsection{Analytic strong norm and Fourier truncation}

\begin{definition}[Analytic strip norm]\label{def:Atau}
Fix $\tau>0$. For $f(x)=\sum_{k\in\mathbb Z}\cHat f(k)e^{2\pi i kx}$ define
\[
  \|f\|_{\mathcal A_\tau}^2
  := \sum_{k\in\mathbb Z} e^{4\pi\tau|k|}\,|\cHat f(k)|^2.
\]
Let $\mathcal A_\tau(\T):=\{f\in L^2(\T):\|f\|_{\mathcal A_\tau}<\infty\}$.
\end{definition}

\begin{lemma}[Exponential tail for Fourier truncation]\label{lem:tail}
Let $\Pi_N$ be the $L^2$-orthogonal projection onto Fourier modes $|k|\le N$.
Then for all $f\in\mathcal A_\tau$,
\[
  \|(I-\Pi_N)f\|_{L^2}\ \le\ e^{-2\pi\tau N}\,\|f\|_{\mathcal A_\tau}.
\]
\end{lemma}

\subsection{Noisy transfer operator with additive Gaussian noise}

\begin{definition}[Periodized Gaussian kernel]\label{def:gauss}
Let $\bar\rho_\sigma$ be the periodized Gaussian on $\T$ with Fourier coefficients
\[
\widehat{\bar\rho_\sigma}(k)=e^{-2\pi^2\sigma^2 k^2}.
\]
Define the convolution (noise) operator $C_\sigma u:=\bar\rho_\sigma*u$.
\end{definition}

\begin{lemma}[Gaussian smoothing into analytic strip]\label{lem:gauss-analytic}
For every $\tau>0$ and $\sigma>0$, $C_\sigma:L^2\to \mathcal A_\tau$ is bounded with
\[
\|C_\sigma u\|_{\mathcal A_\tau}
\le S_{\tau,\sigma}\,\|u\|_{2},
\qquad
S_{\tau,\sigma}:=\sup_{k\in\mathbb Z} \exp\!\bigl(2\pi\tau|k|-2\pi^2\sigma^2k^2\bigr).
\]
In particular, $S_{\tau,\sigma}\le \exp(\tau^2/(2\sigma^2))$ (crude but explicit).
\end{lemma}

\begin{definition}[Noisy operator with shift]\label{def:Pc}
Fix a measurable map $T_0:\T\to\T$ (e.g.\ a quadratic map modulo $1$) and define,
for $c\in\R$,
\[
(P_c u)(x):=\int_{\T}\bar\rho_\sigma\bigl(x-T_0(y)-c\bigr)\,u(y)\,dy.
\]
Define $Q_c:=\partial_c P_c$ (derivative w.r.t.\ the shift).
\end{definition}

\begin{lemma}[Mixed bounds for $P_c$]\label{lem:Pc-mixed-analytic}
For every $c\in\R$ and $u\in L^2(\T)$,
\[
\|P_c u\|_{\mathcal A_\tau}\le S_{\tau,\sigma}\,\|u\|_2,
\qquad
\|P_c u\|_2\le \|u\|_2.
\]
\end{lemma}

\begin{definition}[Shift-derivative operator]\label{def:Qc}
Let $Q_c:=\partial_c P_c$. Then
\[
(Q_c u)(x)=-\int_{\T}\bar\rho_\sigma'\bigl(x-T_0(y)-c\bigr)\,u(y)\,dy,
\]
and in Fourier:
\[
\widehat{Q_cu}(k)=(-2\pi i k)\widehat{\bar\rho_\sigma}(k)e^{-2\pi i kc}
\int_{\T}u(y)e^{-2\pi i kT_0(y)}\,dy.
\]
Define
\[
S^{(1)}_{\tau,\sigma}:=\sup_{k\in\mathbb Z} (2\pi|k|)\exp\!\bigl(2\pi\tau|k|-2\pi^2\sigma^2k^2\bigr),
\qquad
C_J:=\sup_{k\in\mathbb Z}(2\pi|k|)e^{-2\pi^2\sigma^2k^2}.
\]
\end{definition}

\begin{lemma}[Mixed bounds for $Q_c$]\label{lem:Qc-mixed-analytic}
For every $c\in\R$ and $u\in L^2(\T)$,
\[
\|Q_cu\|_{\mathcal A_\tau}\le S^{(1)}_{\tau,\sigma}\|u\|_2,
\qquad
\|Q_cu\|_2\le C_J\|u\|_2.
\]
\end{lemma}

\subsection{Self-consistent coupling with analytic observable}

\begin{assumption}[Coupling functional and regularity]\label{ass:coupling-analytic}
Let $\phi\in L^2(\T)$ (in applications $\phi\in\mathcal A_\tau$) and define
\[
m(f):=\langle \phi,f\rangle_{L^2}.
\]
Let $G:\R\to\R$ be $C^1$ on a relevant interval, and set
\[
c(f):=\delta\,G(m(f)),\qquad
c'(f)[h]=\delta\,G'(m(f))\,\langle\phi,h\rangle_{L^2}.
\]
Denote
\[
L_G:=\sup |G'|\quad\text{on the relevant interval},\qquad
\mathrm{Lip}(G):=\sup_{x\neq y}\frac{|G(x)-G(y)|}{|x-y|}\ \ \text{on that interval}.
\]
\end{assumption}

\begin{definition}[Self-consistent map]\label{def:T}
Define
\[
T(f):=P_{c(f)}f.
\]
\end{definition}

\subsection{Two Fourier discretizations and the parameter mismatch}

\begin{definition}[Two choices of $T_N$]\label{def:TNchoices}
Let $N\in\mathbb N$. Define
\begin{align*}
\textbf{(A) ``semi-discrete coupling'':}\quad
T_N^{A}(f)
&:=\Pi_N\,P_{c(f)}\,\Pi_N f,\\
\textbf{(B) ``fully discrete coupling'':}\quad
T_N^{B}(f)
&:=\Pi_N\,P_{c(\Pi_N f)}\,\Pi_N f.
\end{align*}
\end{definition}

\begin{remark}[Trig polynomial case as a special simplification]\label{rem:polycase}
If $\phi$ is a trigonometric polynomial of degree $K$ and $N\ge K$, then
$\langle \phi,\Pi_N f\rangle=\langle \phi,f\rangle$ for all $f\in L^2$.
Hence $c(\Pi_N f)=c(f)$ and therefore $T_N^A=T_N^B$ exactly (and likewise for their differentials).
In this case all ``parameter mismatch'' terms below vanish.
\end{remark}

\begin{lemma}[Mismatch in $m$ and in $c$ for analytic $\phi$]\label{lem:mismatch}
Assume $\phi\in\mathcal A_\tau$ and $f\in L^2$. Then
\[
|m(f)-m(\Pi_N f)|
=|\langle (I-\Pi_N)\phi,f\rangle|
\le \|(I-\Pi_N)\phi\|_2\,\|f\|_2
\le e^{-2\pi\tau N}\,\|\phi\|_{\mathcal A_\tau}\,\|f\|_2.
\]
Consequently,
\[
|c(f)-c(\Pi_N f)|
\le |\delta|\,\mathrm{Lip}(G)\,e^{-2\pi\tau N}\,\|\phi\|_{\mathcal A_\tau}\,\|f\|_2.
\]
\end{lemma}

\begin{remark}[Why $T_N^B$ is convenient in practice]\label{rem:whichTN}
$T_N^B$ is the genuinely finite-dimensional self-consistent map: all ingredients depend only on $\Pi_N f$.
This is typically what you implement if you cannot evaluate the true residual $T(f)-f$.
When $\phi$ is analytic (not a polynomial), $T_N^A$ and $T_N^B$ differ by an exponentially small mismatch
term controlled by Lemma~\ref{lem:mismatch}.
\end{remark}

\subsection{Exponential discretization bounds for $T$ and the mismatch term}

\begin{lemma}[Exponential discretization error for $T$ vs.\ $T_N^A$]\label{lem:T-TA}
For all $f\in \mathcal A_\tau$,
\[
\|T(f)-T_N^{A}(f)\|_{L^2}
\le e^{-2\pi\tau N}\,\bigl(S_{\tau,\sigma}+1\bigr)\,\|f\|_{\mathcal A_\tau}.
\]
Equivalently,
\[
\|T-T_N^A\|_{\mathcal A_\tau\to L^2}\le e^{-2\pi\tau N}\,(S_{\tau,\sigma}+1).
\]
\end{lemma}

\begin{proof}
Split
\[
T(f)-T_N^A(f)=(I-\Pi_N)P_{c(f)}f+\Pi_NP_{c(f)}(I-\Pi_N)f.
\]
Use Lemma~\ref{lem:tail} and Lemma~\ref{lem:Pc-mixed-analytic} exactly as in the polynomial case.
\end{proof}

\begin{lemma}[Mismatch error between $T_N^A$ and $T_N^B$]\label{lem:TA-TB}
Assume $\phi\in\mathcal A_\tau$ and $f\in L^2$, and Assumption~\ref{ass:coupling-analytic}.
Then
\[
\|T_N^A(f)-T_N^B(f)\|_2
\le |c(f)-c(\Pi_N f)|\;C_J\;\|f\|_2,
\]
and hence, using Lemma~\ref{lem:mismatch},
\[
\|T_N^A(f)-T_N^B(f)\|_2
\le |\delta|\,\mathrm{Lip}(G)\,e^{-2\pi\tau N}\,\|\phi\|_{\mathcal A_\tau}\,C_J\;\|f\|_2^2.
\]
\end{lemma}

\begin{proof}
Write $c_A:=c(f)$ and $c_B:=c(\Pi_N f)$. Then
\[
T_N^A(f)-T_N^B(f)=\Pi_N\bigl(P_{c_A}-P_{c_B}\bigr)\Pi_N f.
\]
By the one-dimensional mean value theorem in the parameter $c$,
for some $\theta\in(0,1)$,
\[
P_{c_A}-P_{c_B}=(c_A-c_B)\,Q_{c_B+\theta(c_A-c_B)}.
\]
Therefore,
\[
\|T_N^A(f)-T_N^B(f)\|_2
\le |c_A-c_B|\,\|Q_{\cdot}\Pi_N f\|_2
\le |c(f)-c(\Pi_N f)|\,C_J\,\|f\|_2,
\]
and the final bound follows from Lemma~\ref{lem:mismatch}.
\end{proof}

\subsection{Differentials $DT$, $DT_N$ and exponential bounds}

\begin{lemma}[Fr\'echet derivative of $T$]\label{lem:DT}
Assume Assumption~\ref{ass:coupling-analytic}. Then $T:\mathcal A_\tau\to L^2$ is Fr\'echet differentiable and
\[
DT(f)[h]=P_{c(f)}h + c'(f)[h]\;Q_{c(f)}f,
\qquad
c'(f)[h]=\delta\,G'(m(f))\,\langle\phi,h\rangle.
\]
Moreover,
\[
|c'(f)[h]|\le |\delta|\,L_G\,\|\phi\|_2\,\|h\|_{\mathcal A_\tau}.
\]
\end{lemma}

\begin{lemma}[Fr\'echet derivatives of $T_N^A$ and $T_N^B$]\label{lem:DTN}
Assume Assumption~\ref{ass:coupling-analytic}. Then $T_N^A,T_N^B:\mathcal A_\tau\to L^2$ are Fr\'echet differentiable and
\begin{align*}
DT_N^{A}(f)[h]
&=
\Pi_N\,P_{c(f)}\,\Pi_N h
\;+\;
c'(f)[h]\;\Pi_N\,Q_{c(f)}\,\Pi_N f,\\
DT_N^{B}(f)[h]
&=
\Pi_N\,P_{c(\Pi_N f)}\,\Pi_N h
\;+\;
c'(\Pi_N f)[\Pi_N h]\;\Pi_N\,Q_{c(\Pi_N f)}\,\Pi_N f.
\end{align*}
\end{lemma}

\begin{remark}[Polynomial case]\label{rem:DTpoly}
If $\phi$ is a trigonometric polynomial of degree $K$ and $N\ge K$, then $c(\Pi_N f)=c(f)$ and
$c'(\Pi_N f)[\Pi_N h]=c'(f)[h]$, hence $DT_N^A(f)=DT_N^B(f)$ exactly.
\end{remark}

\begin{lemma}[Exponential bound for $DT-DT_N^A$]\label{lem:DT-DTNA}
Assume Assumption~\ref{ass:coupling-analytic} and let $f\in\mathcal A_\tau$.
Then for all $h\in\mathcal A_\tau$,
\[
\|(DT(f)-DT_N^{A}(f))[h]\|_2
\le e^{-2\pi\tau N}(S_{\tau,\sigma}+1)\|h\|_{\mathcal A_\tau}
+ |\delta|L_G\|\phi\|_2\;e^{-2\pi\tau N}(S^{(1)}_{\tau,\sigma}+C_J)\;\|f\|_{\mathcal A_\tau}\,\|h\|_{\mathcal A_\tau}.
\]
In particular, on $\{f:\|f\|_{\mathcal A_\tau}\le K\}$,
\[
\sup_{\|f\|_{\mathcal A_\tau}\le K}\|DT(f)-DT_N^{A}(f)\|_{\mathcal A_\tau\to L^2}
\le e^{-2\pi\tau N}(S_{\tau,\sigma}+1)
+ |\delta|L_G\|\phi\|_2\;K\;e^{-2\pi\tau N}(S^{(1)}_{\tau,\sigma}+C_J).
\]
\end{lemma}

\begin{proof}
\textbf{Step 0: remark about the choice of $T_N$.}
We use $T_N(f)=\Pi_NP_{c(f)}\Pi_N f$ (``semi-discrete coupling'').
If instead one uses the fully discrete coupling $T_N^{B}(f)=\Pi_NP_{c(\Pi_N f)}\Pi_N f$,
then since $\psi$ has degree $K$ and $N\ge K$, we have $\langle\psi,\Pi_N f\rangle=\langle\psi,f\rangle$
and thus $c(\Pi_N f)=c(f)$; hence $T_N^B=T_N$ and the same formulas apply.

\medskip
\textbf{Step 1: Fr\'echet derivative of $T$.}
Write $c=c(f)$ and $c_h=c(f+h)$.
Then
\[
T(f+h)-T(f)=P_{c_h}(f+h)-P_c f
= P_c h + (P_{c_h}-P_c)f + (P_{c_h}-P_c)h.
\]
We claim that
\[
(P_{c_h}-P_c)f = (c_h-c)\,Q_c f + o(\|h\|_{\mathcal A_\tau})\quad\text{in }L^2.
\]
Indeed, for fixed $f$, the map $c\mapsto P_c f$ is $C^1$ (differentiate under the integral,
or use the Fourier formula), with derivative $Q_c f$, so by the mean value theorem in one real variable
there exists $\theta\in(0,1)$ such that
\[
P_{c_h}f-P_c f = (c_h-c)\,Q_{c+\theta(c_h-c)} f.
\]
Hence
\[
\|(P_{c_h}-P_c)f-(c_h-c)Q_c f\|_2
\le |c_h-c|\,\|(Q_{c+\theta(c_h-c)}-Q_c)f\|_2 = o(|c_h-c|),
\]
as $c_h\to c$. Since $c(\cdot)$ is Fr\'echet differentiable and thus
$|c_h-c-c'(f)[h]|=o(\|h\|_{\mathcal A_\tau})$, we obtain
\[
(P_{c_h}-P_c)f = c'(f)[h]\,Q_c f + o(\|h\|_{\mathcal A_\tau}).
\]
Finally, $(P_{c_h}-P_c)h$ is a remainder:
using the (local) Lipschitz continuity in $c$ of $P_c$ as an operator $L^2\to L^2$,
\[
\|(P_{c_h}-P_c)h\|_2\le C|c_h-c|\,\|h\|_2 = o(\|h\|_{\mathcal A_\tau}),
\]
because $|c_h-c|=O(\|h\|_2)\le O(\|h\|_{\mathcal A_\tau})$.
Collecting terms yields the derivative formula
\[
DT(f)[h]=P_{c(f)}h + c'(f)[h]\,Q_{c(f)}f.
\]

\medskip
\textbf{Step 2: Fr\'echet derivative of $T_N$.}
Since $T_N(f)=\Pi_N\,P_{c(f)}\,\Pi_N f$ is a composition of bounded linear maps with the scalar $C^1$
functional $c(\cdot)$, the chain rule gives
\[
DT_N(f)[h]=\Pi_N P_{c(f)}\Pi_N h + c'(f)[h]\;\Pi_N Q_{c(f)}\Pi_N f.
\]

\medskip
\textbf{Step 3: Split $DT-DT_N$.}
Let $c=c(f)$. Using the two formulas,
\[
(DT(f)-DT_N(f))[h]
=
\underbrace{\Bigl(P_c-\Pi_NP_c\Pi_N\Bigr)h}_{\mathrm{(I)}}
\;+\;
c'(f)[h]\;\underbrace{\Bigl(Q_c-\Pi_NQ_c\Pi_N\Bigr)f}_{\mathrm{(II)}}.
\]

\medskip
\textbf{Step 4: Bound term (I).}
Decompose
\[
\bigl(P_c-\Pi_NP_c\Pi_N\bigr)h
=
(I-\Pi_N)P_c h \;+\;\Pi_NP_c(I-\Pi_N)h.
\]
For the first part, apply the exponential tail with $g=P_c h$ and then the mixed bound:
\[
\|(I-\Pi_N)P_c h\|_2 \le e^{-2\pi\tau N}\|P_c h\|_{\mathcal A_\tau}
\le e^{-2\pi\tau N}S_{\tau,\sigma}\|h\|_2
\le e^{-2\pi\tau N}S_{\tau,\sigma}\|h\|_{\mathcal A_\tau}.
\]
For the second part, use $\|\Pi_N\|_{2\to 2}=1$ and the $L^2$ bound for $P_c$:
\[
\|\Pi_NP_c(I-\Pi_N)h\|_2 \le \|P_c(I-\Pi_N)h\|_2
\le \|(I-\Pi_N)h\|_2
\le e^{-2\pi\tau N}\|h\|_{\mathcal A_\tau}.
\]
Thus
\[
\|\mathrm{(I)}\|_2\le e^{-2\pi\tau N}(S_{\tau,\sigma}+1)\|h\|_{\mathcal A_\tau}.
\]

\medskip
\textbf{Step 5: Bound term (II).}
Similarly,
\[
(Q_c-\Pi_NQ_c\Pi_N)f
= (I-\Pi_N)Q_c f + \Pi_NQ_c(I-\Pi_N)f.
\]
For the first part,
\[
\|(I-\Pi_N)Q_c f\|_2
\le e^{-2\pi\tau N}\|Q_c f\|_{\mathcal A_\tau}
\le e^{-2\pi\tau N}S^{(1)}_{\tau,\sigma}\|f\|_2
\le e^{-2\pi\tau N}S^{(1)}_{\tau,\sigma}\|f\|_{\mathcal A_\tau}.
\]
For the second part,
\[
\|\Pi_NQ_c(I-\Pi_N)f\|_2
\le \|Q_c(I-\Pi_N)f\|_2
\le C_J\,\|(I-\Pi_N)f\|_2
\le e^{-2\pi\tau N}C_J\|f\|_{\mathcal A_\tau}.
\]
Hence
\[
\|\mathrm{(II)}\|_2\le e^{-2\pi\tau N}(S^{(1)}_{\tau,\sigma}+C_J)\|f\|_{\mathcal A_\tau}.
\]

\medskip
\textbf{Step 6: Control of $|c'(f)[h]|$.}
By Cauchy--Schwarz,
\[
|\langle\psi,h\rangle|\le \|\psi\|_2\|h\|_2\le \|\psi\|_2\|h\|_{\mathcal A_\tau}.
\]
Therefore
\[
|c'(f)[h]|
=|\delta|\;|G'(\langle\psi,f\rangle)|\;|\langle\psi,h\rangle|
\le |\delta|\,L_G\,\|\psi\|_2\,\|h\|_{\mathcal A_\tau}.
\]

\medskip
\textbf{Step 7: Combine.}
Using the split in Step 3, the bound on (I), and the bound on (II) multiplied by $|c'(f)[h]|$,
we obtain exactly:
\[
\|(DT(f)-DT_N(f))[h]\|_2
\le e^{-2\pi\tau N}(S_{\tau,\sigma}+1)\|h\|_{\mathcal A_\tau}
+ |\delta|L_G\|\psi\|_2\;e^{-2\pi\tau N}(S^{(1)}_{\tau,\sigma}+C_J)\;\|f\|_{\mathcal A_\tau}\,\|h\|_{\mathcal A_\tau}.
\]
Taking the supremum over $\|h\|_{\mathcal A_\tau}=1$ yields the operator norm bound, and then
restricting to $\|f\|_{\mathcal A_\tau}\le K$ gives the stated uniform estimate.
\end{proof}

\begin{remark}[From $DT-DT_N^A$ to $DT-DT_N^B$]\label{rem:DTB}
To estimate $DT-DT_N^B$, add the mismatch term
\[
DT_N^A(f)-DT_N^B(f),
\]
which is controlled by $|c(f)-c(\Pi_N f)|$ and $|c'(f)[h]-c'(\Pi_N f)[\Pi_N h]|$.
When $\phi$ is a trig polynomial and $N\ge \deg(\phi)$ this mismatch vanishes exactly.
When $\phi\in\mathcal A_\tau$, the mismatch is exponentially small by Lemma~\ref{lem:mismatch}.
\end{remark}

\begin{lemma}[Explicit mismatch bound for $DT_N^A-DT_N^B$]\label{lem:DTNA-DTNB-explicit}
Assume $\phi\in\mathcal A_\tau$ and define
\[
m(f):=\langle \phi,f\rangle_{L^2},\qquad c(f):=\delta\,G(m(f)),
\qquad c'(f)[h]=\delta\,G'(m(f))\,\langle\phi,h\rangle.
\]
Assume $G\in C^2$ on the relevant range and set
\[
L_G:=\sup|G'|,\qquad \Lip(G):=\sup_{x\neq y}\frac{|G(x)-G(y)|}{|x-y|},\qquad
L_{G'}:=\sup|G''|.
\]
Let $T_N^A(f)=\Pi_NP_{c(f)}\Pi_N f$ and $T_N^B(f)=\Pi_NP_{c(\Pi_N f)}\Pi_N f$.
Let $Q_c=\partial_cP_c$ and $R_c:=\partial_cQ_c=\partial_c^2P_c$ and assume the $L^2$-bounds
\[
\|Q_c u\|_2\le C_J\|u\|_2,
\qquad
\|R_c u\|_2\le C_J^{(2)}\|u\|_2
\quad\text{uniformly in }c\in\R.
\]
Then for every $f\in L^2(\T)$ and every $h\in\mathcal A_\tau$,
\[
\bigl\|(DT_N^A(f)-DT_N^B(f))[h]\bigr\|_2
\;\le\; e^{-2\pi\tau N}\,\mathcal C(f)\,\|h\|_{\mathcal A_\tau},
\]
where
\[
\mathcal C(f)
:=
|\delta|\,\|\phi\|_{\mathcal A_\tau}\Bigl[
\Lip(G)\,C_J\,\|f\|_2
+
L_G\,C_J\,\|f\|_2
+
L_{G'}\,\|\phi\|_2\,C_J\,\|f\|_2^2
\Bigr]
+
|\delta|^2\,L_G\,\Lip(G)\,\|\phi\|_2\,\|\phi\|_{\mathcal A_\tau}\,C_J^{(2)}\,\|f\|_2^2.
\]
Consequently, on the ball $\{f:\|f\|_{\mathcal A_\tau}\le K\}$ (hence $\|f\|_2\le K$),
\[
\sup_{\|f\|_{\mathcal A_\tau}\le K}
\|DT_N^A(f)-DT_N^B(f)\|_{\mathcal A_\tau\to L^2}
\;\le\;
e^{-2\pi\tau N}\Bigl(
|\delta|\,\|\phi\|_{\mathcal A_\tau}\bigl[(\Lip(G)+L_G)C_JK
+L_{G'}\|\phi\|_2C_JK^2\bigr]
+|\delta|^2L_G\Lip(G)\|\phi\|_2\|\phi\|_{\mathcal A_\tau}C_J^{(2)}K^2
\Bigr).
\]

\medskip
\noindent\textbf{Polynomial case.}
If $\phi$ is a trigonometric polynomial of degree $K_0$ and $N\ge K_0$, then
$m(\Pi_N f)=m(f)$ for all $f$, so $c(\Pi_N f)=c(f)$ and $c'(\Pi_N f)[\Pi_N h]=c'(f)[h]$.
Hence $DT_N^A(f)=DT_N^B(f)$ exactly for all $f,h$.
\end{lemma}

\begin{proof}
Write $c_A:=c(f)$ and $c_B:=c(\Pi_N f)$ and
\[
\alpha:=c'(f)[h]=\delta\,G'(m(f))\langle\phi,h\rangle,\qquad
\beta:=c'(\Pi_N f)[\Pi_N h]=\delta\,G'(m(\Pi_N f))\langle\phi,\Pi_N h\rangle.
\]
Using the formulas for the Fr\'echet derivatives,
\[
DT_N^A(f)[h]=\Pi_NP_{c_A}\Pi_N h+\alpha\,\Pi_NQ_{c_A}\Pi_N f,
\qquad
DT_N^B(f)[h]=\Pi_NP_{c_B}\Pi_N h+\beta\,\Pi_NQ_{c_B}\Pi_N f.
\]
Hence
\begin{align*}
(DT_N^A-DT_N^B)[h]
&=\Pi_N(P_{c_A}-P_{c_B})\Pi_N h
+(\alpha-\beta)\,\Pi_NQ_{c_A}\Pi_N f
+\beta\,\Pi_N(Q_{c_A}-Q_{c_B})\Pi_N f.
\end{align*}
We bound the three terms separately in $L^2$ (using $\|\Pi_N\|_{2\to2}=1$).

\smallskip
\noindent\textbf{(I) The $P$-mismatch.}
By the one-dimensional mean value theorem in the parameter $c$,
\[
P_{c_A}-P_{c_B}=(c_A-c_B)\,Q_{c_\theta}
\quad\text{for some }c_\theta=c_B+\theta(c_A-c_B).
\]
Thus
\[
\|\Pi_N(P_{c_A}-P_{c_B})\Pi_N h\|_2
\le |c_A-c_B|\,\|Q_{c_\theta}\Pi_N h\|_2
\le |c_A-c_B|\,C_J\,\|h\|_2
\le |c_A-c_B|\,C_J\,\|h\|_{\mathcal A_\tau}.
\]

\smallskip
\noindent\textbf{(II) The $c'$-mismatch.}
We estimate
\[
|\alpha-\beta|
\le |\delta|\,|G'(m(f))|\,|\langle\phi,h-\Pi_N h\rangle|
+|\delta|\,|\langle\phi,\Pi_N h\rangle|\,|G'(m(f))-G'(m(\Pi_N f))|.
\]
Since $\phi\in\mathcal A_\tau$,
\[
|\langle\phi,h-\Pi_N h\rangle|
=|\langle (I-\Pi_N)\phi,h\rangle|
\le \|(I-\Pi_N)\phi\|_2\,\|h\|_2
\le e^{-2\pi\tau N}\|\phi\|_{\mathcal A_\tau}\,\|h\|_{\mathcal A_\tau}.
\]
Also $|\langle\phi,\Pi_N h\rangle|\le \|\phi\|_2\|h\|_2\le \|\phi\|_2\|h\|_{\mathcal A_\tau}$.
Finally, by Lipschitz continuity of $G'$ (i.e.\ bounded $G''$),
\[
|G'(m(f))-G'(m(\Pi_N f))|
\le L_{G'}\,|m(f)-m(\Pi_N f)|
=L_{G'}\,|\langle (I-\Pi_N)\phi,f\rangle|
\le L_{G'}\,e^{-2\pi\tau N}\|\phi\|_{\mathcal A_\tau}\,\|f\|_2.
\]
Combining gives
\[
|\alpha-\beta|
\le e^{-2\pi\tau N}|\delta|\,\|\phi\|_{\mathcal A_\tau}\Bigl[
L_G + L_{G'}\|\phi\|_2\,\|f\|_2
\Bigr]\|h\|_{\mathcal A_\tau}.
\]
Therefore
\[
\|(\alpha-\beta)\Pi_NQ_{c_A}\Pi_N f\|_2
\le |\alpha-\beta|\,C_J\,\|f\|_2
\le e^{-2\pi\tau N}|\delta|\,\|\phi\|_{\mathcal A_\tau}\Bigl[
L_G\,C_J\,\|f\|_2 + L_{G'}\|\phi\|_2\,C_J\,\|f\|_2^2
\Bigr]\|h\|_{\mathcal A_\tau}.
\]

\smallskip
\noindent\textbf{(III) The $Q$-mismatch.}
Again by the mean value theorem,
\[
Q_{c_A}-Q_{c_B}=(c_A-c_B)\,R_{c_{\theta'}}.
\]
Hence
\[
\|\beta\,\Pi_N(Q_{c_A}-Q_{c_B})\Pi_N f\|_2
\le |\beta|\,|c_A-c_B|\,\|R_{c_{\theta'}}\Pi_N f\|_2
\le |\beta|\,|c_A-c_B|\,C_J^{(2)}\,\|f\|_2.
\]
Moreover $|\beta|\le |\delta|\,L_G\,\|\phi\|_2\,\|h\|_2\le |\delta|L_G\|\phi\|_2\|h\|_{\mathcal A_\tau}$.
And (by Lipschitz continuity of $G$ and the same tail estimate)
\[
|c_A-c_B|
=|\delta|\,|G(m(f))-G(m(\Pi_N f))|
\le |\delta|\,\Lip(G)\,|m(f)-m(\Pi_N f)|
\le |\delta|\,\Lip(G)\,e^{-2\pi\tau N}\|\phi\|_{\mathcal A_\tau}\,\|f\|_2.
\]
Thus
\[
\|\beta\,\Pi_N(Q_{c_A}-Q_{c_B})\Pi_N f\|_2
\le e^{-2\pi\tau N}\,|\delta|^2\,L_G\,\Lip(G)\,\|\phi\|_2\,\|\phi\|_{\mathcal A_\tau}\,C_J^{(2)}\,\|f\|_2^2\,\|h\|_{\mathcal A_\tau}.
\]

\smallskip
\noindent\textbf{Conclusion.}
Summing (I)--(III) gives the displayed bound with $\mathcal C(f)$.
The uniform bound on $\|f\|_{\mathcal A_\tau}\le K$ follows from $\|f\|_2\le K$.
Finally, in the polynomial case with $\deg(\phi)\le N$ we have $m(\Pi_N f)=m(f)$ and
$\langle\phi,\Pi_N h\rangle=\langle\phi,h\rangle$, hence $c_A=c_B$ and $\alpha=\beta$, so the difference is $0$.
\end{proof}

% % ============================================================
% Corollary: NK for the TRUE self-consistent operator T in mixed norm,
% certified from a fully discrete Fourier surrogate T_N.
% Strong space: A_tau, weak space: L^2.
% ============================================================

\begin{corollary}[Mixed Newton--Kantorovich for $T$ via Fourier surrogate $T_N$]\label{cor:NK-mixed-true-T}
Fix $\tau>0$, and set $\mathcal B_s:=\mathcal A_\tau(\T)$ and $\mathcal B_w:=L^2(\T)$
(with $\|u\|_2\le \|u\|_{\mathcal A_\tau}$).
Let $\Pi_N$ be the $L^2$-orthogonal projection onto Fourier modes $|k|\le N$.

Let $\phi\in\mathcal A_\tau$ and define the coupling
\[
m(f):=\langle \phi,f\rangle_{L^2},\qquad c(f):=\delta\,G(m(f)),
\]
where $G\in C^2$ on the relevant range. Define the true self-consistent operator
\[
T(f):=P_{c(f)}f,
\]
where $P_c$ is the noisy (periodized Gaussian) shift-kernel operator, and set
\[
T_N(f):=T_N^B(f):=\Pi_N\,P_{c(\Pi_N f)}\,\Pi_N f
\qquad\text{(fully discrete self-consistent map)}.
\]
Let $\ell\in(\mathcal B_w)^*$ be the normalization functional and consider the constrained
fixed-point problem $T(f)=f$, $\ell(f)=1$; Newton is performed on $\ker\ell$.

Assume the following uniform bounds (``LY-w'' / ``LY-s'' style hypotheses):
\begin{align*}
\text{(P-w)}\quad &\|P_c u\|_2\le \|u\|_2,\\
\text{(P-s)}\quad &\|P_c u\|_{\mathcal A_\tau}\le S_{\tau,\sigma}\,\|u\|_2,\\
\text{(Q-w)}\quad &\|Q_c u\|_2\le C_J\,\|u\|_2,\qquad Q_c:=\partial_c P_c,\\
\text{(R-w)}\quad &\|R_c u\|_2\le C_J^{(2)}\,\|u\|_2,\qquad R_c:=\partial_c^2 P_c,
\end{align*}
and the analytic Fourier tail estimate
\[
\|(I-\Pi_N)g\|_2\le e^{-2\pi\tau N}\,\|g\|_{\mathcal A_\tau}.
\]
Denote the coupling constants on the relevant range
\[
L_G:=\sup|G'|,\qquad \mathrm{Lip}(G):=\sup_{x\neq y}\frac{|G(x)-G(y)|}{|x-y|},\qquad L_{G'}:=\sup|G''|.
\]

Let $\tilde f\in\mathrm{Ran}(\Pi_N)$ be a candidate with $\ell(\tilde f)=1$. Define the
\emph{computable} quantities
\[
\delta_N:=\|T_N(\tilde f)-\tilde f\|_2,\qquad
J_N:=\bigl(I-DT_N(\tilde f)\bigr)\big|_{\mathrm{Ran}(\Pi_N)\cap\ker\ell},\qquad
M_N:=\|J_N^{-1}\|_{2\to 2},
\]
and the a priori size bounds
\[
K_2:=\|\tilde f\|_2,\qquad K_\tau:=\|\tilde f\|_{\mathcal A_\tau}.
\]

\smallskip
\noindent\textbf{(1) Explicit residual bound for the true map.}
Set the \emph{true residual}
\[
\Delta:=\|T(\tilde f)-\tilde f\|_2.
\]
Then
\[
\Delta \le \delta_N + e_T + e_{\mathrm{mis}},
\]
where
\[
e_T := e^{-2\pi\tau N}(S_{\tau,\sigma}+1)\,K_\tau,
\qquad
e_{\mathrm{mis}} := |\delta|\,\mathrm{Lip}(G)\,e^{-2\pi\tau N}\,\|\phi\|_{\mathcal A_\tau}\,C_J\,K_2^2.
\]
(In particular, if $\phi$ is a trigonometric polynomial of degree $\le N$, then
$m(\Pi_N f)=m(f)$ and $e_{\mathrm{mis}}=0$.)

\smallskip
\noindent\textbf{(2) Last explicit bound: Jacobian discretization error.}
Define the Jacobian mismatch
\[
\varepsilon_J:=\bigl\|DT(\tilde f)-DT_N(\tilde f)\bigr\|_{\mathcal A_\tau\to L^2}.
\]
Then one may take
\[
\varepsilon_J \ \le\ e^{-2\pi\tau N}(S_{\tau,\sigma}+1)
\;+\;|\delta|L_G\|\phi\|_2\,K_\tau\,e^{-2\pi\tau N}\bigl(S^{(1)}_{\tau,\sigma}+C_J\bigr)
\;+\; e^{-2\pi\tau N}\,C_{\mathrm{mis}}(K_2),
\]
where the (explicit) mismatch constant can be chosen as
\[
C_{\mathrm{mis}}(K_2):=
|\delta|\,\|\phi\|_{\mathcal A_\tau}\Bigl[(\mathrm{Lip}(G)+L_G)C_JK_2
+L_{G'}\|\phi\|_2\,C_JK_2^2\Bigr]
+|\delta|^2\,L_G\mathrm{Lip}(G)\,\|\phi\|_2\,\|\phi\|_{\mathcal A_\tau}\,C_J^{(2)}K_2^2.
\]
(Again, if $\deg(\phi)\le N$ then $C_{\mathrm{mis}}(K_2)=0$.)

\smallskip
\noindent\textbf{(3) Certified inverse bound for the true Jacobian on $\ker\ell$.}
If
\[
M_N\,\varepsilon_J < 1,
\]
then $J(\tilde f):=(I-DT(\tilde f))|_{\ker\ell}$ is invertible and
\[
M:=\|J(\tilde f)^{-1}\|_{2\to 2;\ker\ell}\ \le\ \frac{M_N}{1-M_N\varepsilon_J}.
\]

\smallskip
\noindent\textbf{(4) Lipschitz constant for the differential (mixed norm).}
On any $L^2$-ball $\{f:\|f\|_2\le K\}$, one may take the (explicit) Lipschitz constant
\[
\|DT(f)-DT(g)\|_{2\to 2}\ \le\ \gamma(K)\,\|f-g\|_2,
\]
with
\[
\gamma(K):=
|\delta|\,\|\phi\|_2\,C_J\bigl(\mathrm{Lip}(G)+L_G\bigr)
\;+\;|\delta|\,L_{G'}\,\|\phi\|_2^2\,C_J\,K
\;+\;|\delta|^2\,L_G\mathrm{Lip}(G)\,\|\phi\|_2^2\,C_J^{(2)}\,K.
\]
(For the Kantorovich test below you may take any $K\ge \|\tilde f\|_2+r$; in practice one
often starts with $K:=\|\tilde f\|_2$ and enlarges if needed.)

\smallskip
\noindent\textbf{(5) Kantorovich smallness and conclusion.}
Let $K\ge \|\tilde f\|_2$ and define
\[
h:= M^2\,\gamma(K)\,\Delta.
\]
Assume
\[
h\le \frac12.
\]
Then there exists a \emph{unique} normalized fixed point $f^\ast$ of $T$ with $\ell(f^\ast)=1$ in
the $L^2$-ball $B_r(\tilde f)$, and
\[
\|f^\ast-\tilde f\|_2\le r,
\qquad
r:=\frac{1-\sqrt{1-2h}}{M\,\gamma(K)}
\quad\text{(interpreting $r:=M\Delta$ if $\gamma(K)=0$)}.
\]
Moreover, the Newton map on $\ell(f)=1$ is a contraction on $B_r(\tilde f)$ with contraction
constant
\[
\kappa:=\frac{1-\sqrt{1-2h}}{h}\;<\;1.
\]

\smallskip
\noindent\textbf{What must be computed numerically.}
To apply the corollary you need only:
\begin{enumerate}
\item the Fourier candidate $\tilde f\in\mathrm{Ran}(\Pi_N)$ with $\ell(\tilde f)=1$;
\item the finite-dimensional residual $\delta_N=\|T_N(\tilde f)-\tilde f\|_2$;
\item the finite-dimensional Jacobian $J_N=(I-DT_N(\tilde f))$ on $\mathrm{Ran}(\Pi_N)\cap\ker\ell$
and a certified bound $M_N=\|J_N^{-1}\|_{2\to 2}$;
\item the norms $K_2=\|\tilde f\|_2$ and $K_\tau=\|\tilde f\|_{\mathcal A_\tau}$ (both explicit from Fourier coefficients).
\end{enumerate}
All remaining constants are explicit from $(\sigma,\tau,\delta,G,\phi)$ and the Gaussian bounds
$S_{\tau,\sigma}$, $C_J$, $C_J^{(2)}$.
\end{corollary}
% ============================================================
% Explicit Gaussian constants S_{tau,sigma}, S^{(1)}, S^{(2)}
% for periodized Gaussian noise (Fourier multiplier).
% ============================================================

\begin{lemma}[Explicit Gaussian smoothing constants]\label{lem:S-constants-gaussian}
Let $\bar\rho_\sigma$ be the periodized Gaussian on $\T$ with
\[
\widehat{\bar\rho_\sigma}(k)=e^{-2\pi^2\sigma^2k^2},\qquad k\in\mathbb Z.
\]
Define, for $\tau\ge 0$,
\begin{align*}
S_{\tau,\sigma}
&:=\sup_{k\in\mathbb Z}\exp\!\bigl(2\pi\tau|k|-2\pi^2\sigma^2k^2\bigr),\\
S^{(1)}_{\tau,\sigma}
&:=\sup_{k\in\mathbb Z}(2\pi|k|)\exp\!\bigl(2\pi\tau|k|-2\pi^2\sigma^2k^2\bigr),\\
S^{(2)}_{\tau,\sigma}
&:=\sup_{k\in\mathbb Z}(2\pi|k|)^2\exp\!\bigl(2\pi\tau|k|-2\pi^2\sigma^2k^2\bigr).
\end{align*}
Then:
\begin{enumerate}
\item (\textbf{Exact computability}) Each supremum is attained at a finite $k$ and can be computed
      by checking $k$ in a small neighborhood of the continuous maximizer:
      \[
      k_0:=\Bigl\lfloor \frac{\tau}{2\pi\sigma^2}\Bigr\rceil
      \quad\Longrightarrow\quad
      S_{\tau,\sigma}=\max\{f(k_0-1),f(k_0),f(k_0+1)\},
      \]
      where $f(k)=\exp(2\pi\tau|k|-2\pi^2\sigma^2k^2)$.
      Similarly $S^{(1)}_{\tau,\sigma}$ and $S^{(2)}_{\tau,\sigma}$ are attained near the
      continuous maximizers of $k\mapsto (2\pi k)^j e^{2\pi\tau k-2\pi^2\sigma^2k^2}$.

\item (\textbf{Fully explicit bounds}) One has the closed-form upper bounds
      \[
      S_{\tau,\sigma}\le \exp\!\Bigl(\frac{\tau^2}{2\sigma^2}\Bigr),
      \]
      \[
      S^{(1)}_{\tau,\sigma}\le \exp\!\Bigl(\frac{\tau^2}{2\sigma^2}\Bigr)
      \Bigl(\frac{\tau}{\sigma^2}+\frac{1}{\sigma\sqrt{e}}\Bigr),
      \]
      \[
      S^{(2)}_{\tau,\sigma}\le \exp\!\Bigl(\frac{\tau^2}{2\sigma^2}\Bigr)
      \Bigl(\frac{\tau^2}{\sigma^4}+\frac{2\tau}{\sigma^3\sqrt{e}}+\frac{2}{\sigma^2 e}\Bigr).
      \]
      In particular (take $\tau=0$),
      \[
      C_J\le \frac{1}{\sigma\sqrt{e}},
      \qquad
      C_J^{(2)}\le \frac{2}{\sigma^2 e}.
      \]
\end{enumerate}
\end{lemma}

\begin{proof}
For $S_{\tau,\sigma}$, complete the square:
\[
2\pi\tau k-2\pi^2\sigma^2k^2
=
-2\pi^2\sigma^2\Bigl(k-\frac{\tau}{2\pi\sigma^2}\Bigr)^2+\frac{\tau^2}{2\sigma^2},
\]
so $\sup_k e^{2\pi\tau k-2\pi^2\sigma^2k^2}\le e^{\tau^2/(2\sigma^2)}$.
For $S^{(1)}_{\tau,\sigma}$ write $k=b+y$ with $b=\tau/(2\pi\sigma^2)$ and use
$k\le b+|y|$ together with
\[
\sup_{y\in\mathbb R}|y|e^{-a y^2}=\frac{1}{\sqrt{2ae}},
\qquad a=2\pi^2\sigma^2,
\]
which yields the stated bound after multiplying by $2\pi e^{\tau^2/(2\sigma^2)}$.
For $S^{(2)}_{\tau,\sigma}$ use $k^2\le b^2+2b|y|+y^2$ and
\[
\sup_{y\in\mathbb R}y^2e^{-a y^2}=\frac{1}{ae}.
\]
The ``exact computability'' follows because all these sequences are unimodal for $k\ge 0$
and decay super-exponentially in $k$.
\end{proof}

\end{document}
